% TODO Anhänge anfügen
% Wir haben dies jeweils über \chapter gelöst
% \includepdf[pages=-]{PDF-ANHANG}
\chapter{Supplementary Material}
\label{ch:Supplementary}

% Brightness change

\begin{figure*}
    \centering
    \setlength{\tabcolsep}{1pt}
    \Large
    \resizebox{\textwidth}{!}{%
        \begin{tabular}{C{5em}ccccc}
            & Level 1 & Level 2 & Level 3 & Level 4 & Level 5 \\
            Brighten & \includegraphics[width=\gridimagewidth,valign=m]{img/Blur.jpg} & \includegraphics[width=\gridimagewidth,valign=m]{img/Artifacts.jpg} & \includegraphics[width=\gridimagewidth,valign=m]{img/Artifacts.jpg} & \includegraphics[width=\gridimagewidth,valign=m]{img/Artifacts.jpg} & \includegraphics[width=\gridimagewidth,valign=m]{img/Artifacts.jpg} \\ [6.15ex]
            Darken & \includegraphics[width=\gridimagewidth,valign=m]{img/Artifacts.jpg} & \includegraphics[width=\gridimagewidth,valign=m]{img/Artifacts.jpg} & \includegraphics[width=\gridimagewidth,valign=m]{img/Artifacts.jpg} & \includegraphics[width=\gridimagewidth,valign=m]{img/Artifacts.jpg} & \includegraphics[width=\gridimagewidth,valign=m]{img/Artifacts.jpg} \\ [6.15ex]
        \end{tabular}
    }
    \caption{Visualization of the degradation types belonging to the \textit{Brightness change} group for increasing levels of intensity.}
    \label{fig:brightness_change_supplementary}
\end{figure*}


\begin{comment}
% Blur
\begin{figure*}
    \centering
    \setlength{\tabcolsep}{1pt}
    \Large
    \resizebox{\textwidth}{!}{ %< auto-adjusts font size to fill line
\begin{tabular}{C{6em}ccccc}
         & Level 1 & Level 2 & Level 3 & Level 4 & Level 5 \\
 Gaussian blur & \includegraphics[width=\gridimagewidth,valign=m]{images/supplementary/distortions/blur/gaublur0.jpg} & \includegraphics[width=\gridimagewidth,valign=m]{images/supplementary/distortions/blur/gaublur1.jpg} & \includegraphics[width=\gridimagewidth,valign=m]{images/supplementary/distortions/blur/gaublur2.jpg} & \includegraphics[width=\gridimagewidth,valign=m]{images/supplementary/distortions/blur/gaublur3.jpg} & \includegraphics[width=\gridimagewidth,valign=m]{images/supplementary/distortions/blur/gaublur4.jpg} \\ [6.15ex]
 Lens blur & \includegraphics[width=\gridimagewidth,valign=m]{images/supplementary/distortions/blur/lensblur0.jpg} & \includegraphics[width=\gridimagewidth,valign=m]{images/supplementary/distortions/blur/lensblur1.jpg} & \includegraphics[width=\gridimagewidth,valign=m]{images/supplementary/distortions/blur/lensblur2.jpg} & \includegraphics[width=\gridimagewidth,valign=m]{images/supplementary/distortions/blur/lensblur3.jpg} & \includegraphics[width=\gridimagewidth,valign=m]{images/supplementary/distortions/blur/lensblur4.jpg} \\ [6.15ex]
 Motion blur & \includegraphics[width=\gridimagewidth,valign=m]{images/supplementary/distortions/blur/motionblur0.jpg} & \includegraphics[width=\gridimagewidth,valign=m]{images/supplementary/distortions/blur/motionblur1.jpg} & \includegraphics[width=\gridimagewidth,valign=m]{images/supplementary/distortions/blur/motionblur2.jpg} & \includegraphics[width=\gridimagewidth,valign=m]{images/supplementary/distortions/blur/motionblur3.jpg} & \includegraphics[width=\gridimagewidth,valign=m]{images/supplementary/distortions/blur/motionblur4.jpg} \\ 
\end{tabular}
}
\caption{Visualization of the degradation types belonging to the \textit{Blur} group for increasing levels of intensity.}
\label{fig:blur_supplementary}
\end{figure*}
\end{comment}


\chapter{Code}
\label{ch:Code}
Anhang, Abkürzungs-, Abbildungs-, Tabellen-, Formel-Verzeichnis, Literaturverzeichnis nicht vergessen!\par
\textbf{Anhänge}

Projektspezifisch können weitere Dokumentationsteile angefügt werden wie:

Aufgabenstellung, Projektmanagement-Plan/Bericht, Testplan/Testbericht, Bedienungsanleitungen, Details zu Umfragen, detaillierte Anforderungslisten, Referenzen auf projektspezifische Daten in externen Entwicklungs- und Datenverwaltungstools etc.
\begin{lstlisting}[caption={Caption on PDF}, label={lst:reference_this}, language=Python]
import numpy as np
\end{lstlisting}