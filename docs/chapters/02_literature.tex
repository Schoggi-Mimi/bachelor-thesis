\chapter{Literature Review}
\label{ch:LiteratureReview}
text

\section{Image Quality Assessment (IQA)}
\label{sec:OverviewTeledermatology}
text \par
\vspace{\baselineskip}
\noindent

\subsection{Introduction to IQA}
\label{sub:IntroductionIQA}
text \par
\vspace{\baselineskip}
\noindent

\subsection{Metrics Used in IQA}
\label{sub:MetricsIQA}
text \par
\vspace{\baselineskip}
\noindent

\subsection{Benchmark Datasets for IQA}
\label{sub:BenchmarkDatasetsIQA}
text \par
\vspace{\baselineskip}
\noindent

\subsection{State-of-the-Art in IQA}
\label{sub:SOTA_IQA}
text \par
\vspace{\baselineskip}
\noindent

\subsection{Quality Criteria for Image Assessment}
\label{sub:QualityCriteria}
text \par
\vspace{\baselineskip}
\noindent

\subsection{Challenges and Opportunities in IQA}
\label{sub:ChallengesOpportunitiesIQA}
text \par
\vspace{\baselineskip}
\noindent

\subsection{Previous Research in IQA}
\label{sub:PreviousResearchIQA}
text \par
\vspace{\baselineskip}
\noindent



\section{Teledermatology}
\label{sec:Teledermatology}
The following section provides an overview of teledermatology, a specialized field of dermatology that utilizes telecommunications technology to provide remote diagnosis and consultation for skin conditions. This section discusses the importance of image quality in teledermatology, quality criteria for teledermatology images, as well as challenges and opportunities associated with the practice. \par
\vspace{\baselineskip}
\noindent

\subsection{Introduction to Teledermatology}
\label{sub:IntroductionTeledermatology}
The term "teledermatology" combines "tele", which refers to distance or remote communication, and "dermatology", the medical field focused on skin health. This specialized branch of dermatolgoy utilizes telecommunications technology to provide remote diagnosis and consultation for skin conditions.\par
\vspace{\baselineskip}
\noindent
This innovative approach to healthcare delivery is particularly beneficial for patients in remote or underserved areas, as well as for those with mobility issues. Teledermatology services can be provided in real-time, or through store and forward images, wherein the patient captures images of their skin or any skin-related issues using a camera or smartphone and send them electronically to a dermatologist, along with relevant details about their condition, such as symptoms and medical history. This allows dermatologists to assess the skin condition remotely and provide recommendations or treatment plans without the need for an in-person visit.\par

\subsection{Importance of Image Quality in Teledermatology}
\label{sub:ImportanceIQA_Teledermatology}
High-quality images are essential for accurate diagnosis in teledermatology. While poor image quality can lead to misinterpretation of skin lesions, incorrect diagnosis or missed diagnosis. \par
\vspace{\baselineskip}
\noindent
With good image quality the dermatologists can better assess the severity of skin conditions and formulate appropriate treatment plans.\par
\vspace{\baselineskip}
\noindent
No in-persons visits and improve accessibility to specialized care.\par
\vspace{\baselineskip}
\noindent
Maintaining consistent image quality standards ensures the reliablility and reproducibility of teledermatology services. It minimizes variability and enhances the overall reliability of remote diagnosis and consultation process\par
\vspace{\baselineskip}
\noindent
show good and bad quality images!!\par

\subsection{Quality Criteria for Teledermatology Images}
\label{sub:QualityCriteriaTeledermatology}
The table is temporary and will be replaced as subsubsections with more detailed information and images later on. \par
\begin{tabular}{|p{0.26\linewidth}|p{0.68\linewidth}|}
\hline
\textbf{Criteria} & \textbf{Description} \\
\hline
Lighting & - make sure skin color/tone are accurately captured.\par - AVOID using flash or light: could whiten skin tone, reduce contrast and cause reflection.\par - USE natural light: best for regional and close up, but impractical in clinical settings.\par \textbf{position light source at an angle to skin (not directly overhead or perpendicular)}\\
\hline
Background Color & - reflection from object in background can change appereance of skin color. \par \textbf{solid background color and contrast between background and skin} \\
\hline
Field of View for \par Dermoscopic Images & - enough distance from skin to include entire lesion. \par - multiple images help ensure all edges of lesion are visualized and recorded \par \textbf{center lesion of area of interest} \\
\hline
Image Orientation & - consistency is important to compare area of interest over time. \par - cephalic, vertical, horizontal orientation \\
\hline
Focus and Depth of Field & - camera perpendicular to skin and lens with deep depth of field \par \textbf{center of lesoin or area of interest should be used as focus point} \\
\hline
Resolution & - defines how much detail to capture and result in larger file size \par - hair follicles should be sharp in regional images \par - skin markings should be sharp visible in close up images \par \textbf{JPEG and at least 200KB in size} \\
\hline
Scale and Measurement & - report lesion size and changes in dimension over time \par - avoid problem of skewed rulers \par \textbf{digital scale incoperated into devices and software} \\
\hline
Color Calibration & - should be comparable over time and regulary calibrated.\\
\hline
Image Storage & - store for regulatory and clinical reasons \par \textbf{JPEG, TIFF, EXIF, DICOM} \\
\hline
\end{tabular}


\subsection{Challenges and Opportunities in Teledermatology}
\label{sub:ChallengesOpportunitiesTeledermatology}
Challenges: picture taken by the patient is not in a good qualtiy, patient data security and privacy, including compliance with regulations. The whole patient cannot be examined, only localised. No touching of skin. Demands diligence in documentation, storage and consent. Who has the clinical accountability or responsibility. Double charging. Teledermatology is not included in training curriculum for doctors. Different patient experience. Barriers in practice such as individual preference of doctors, resistance to change and no benefit in investing time to adapt. \par
\vspace{\baselineskip}
\noindent
Opportunities: increase access to care, reduce waiting times, reduce travel time and costs, increase patient satisfaction, increase efficiency, increase access to specialist care, increase access to education and training, increase access to research and clinical trials, increase access to data and analytics, increase access to technology and innovation, increase access to collaboration and networking, increase access to telemedicine and telehealth. \par

\subsection{Previous Research in Teledermatology}
\label{sub:PreviousResearchTeledermatology}
The article by Primary Care Commissioning in 2011 outlined quality standards for teledermatology services using store and forward images. These standards include: \par
\noindent
\begin{itemize}
    \item Standard 1: Models of teledermatology services including links to other services
    \item Standard 2: Selecting patients for teledermatology
    \item Standard 3: Gaining the patient's informed consent
    \item Standard 4: Competent staff
    \item Standard 5: The teledermatology referral: patient history and suitable images
    \item Standard 6: Communication between referring and reporting clinician
    \item Standard 7: Information governance and record-keeping
    \item Standard 8: Audit and quality control
\end{itemize}
\par
\noindent
These standards serve as guidelines for ensuring the quality and effectiveness of teledermatology services, particularly in the context of using store and forward images.