\chapter{Introduction}
\label{ch:Introduction}
Problem, Fragestellung, Vision

Welche Ziele, Fragestellungen werden mit dem Projekt verfolgt? Die Bedeutung, Auswirkung und Relevanz dieses Projektes für die unterschiedlichen Beteiligten soll aufgeführt werden.

Typischerweise wird hier ein Verweis auf die Aufgabenstellung im Anhang gemacht. \par
\vspace{\baselineskip}
\noindent
Teledermatological consultations have become much more common lately in the past two years because of the SARS CoV-2 (COVID-19) pandameic. The consultations are typically done via applicaitons where the patients take a photograph of their skin lesions using their mobile devices, such as smartphones and tablets, and send them to their dermatologists for advice. However, sometimes the pictures are not very good because of things like bad lighting, being blurry, or not showing the skin problem clearly. This can make it harder for the dermatologist to give the right diagnosis. So, it's important to try and take clear and well-lit pictures when using these apps for skin consultations.
\par
\vspace{\baselineskip}
\noindent

\section{Background and Problem Statement}
\label{sec:BackgroundProblemStatement}
introduce importance of automated IQA, particularly in teledermatology \par 
highlight recent advancements in AI and its applicaiton in image analysis, including role of deep learning \par
identify challenges associated with manual image quality asessment in TD and the impact of poor image quality on diagnosis accuracy \par

\vspace{\baselineskip}
\noindent

\section{Objectives of the Thesis}
\label{sec:Objectives}
primary objective of your thesis, which is to develop and evaluate automated image quality assessment methods for teledermatology \par 
specific goals, such as conducting a literature review, selecting appropriate quality assessment metrics, evaluating methods on relevant datasets, and developing a reproducible repository. \par
scope of your thesis, including the focus on evaluating existing methods, adapting them to the teledermatology context, and proposing potential improvements. \par
\vspace{\baselineskip}
\noindent

\section{Structure of the Thesis}
\label{sec:Structure}
text \par
\vspace{\baselineskip}
\noindent
