\chapter{Introduction}
\label{ch:Introduction}
Problem, Fragestellung, Vision

Welche Ziele, Fragestellungen werden mit dem Projekt verfolgt? Die Bedeutung, Auswirkung und Relevanz dieses Projektes für die unterschiedlichen Beteiligten soll aufgeführt werden.

Typischerweise wird hier ein Verweis auf die Aufgabenstellung im Anhang gemacht. \par
\vspace{\baselineskip}
\noindent
introduce importance of automated IQA, particularly in teledermatology \par 
highlight recent advancements in AI and its applicaiton in image analysis, including role of deep learning \par
identify challenges associated with manual image quality asessment in TD and the impact of poor image quality on diagnosis accuracy \par
\par
\vspace{\baselineskip}
\noindent

\section{Background and Problem Statement}
\label{sec:BackgroundProblemStatement}
In recent years, the way we seek dermatological advice has changed significantly, mainly due to the COVID-19 pandemic. Teledermatology, a branch of telemedicine, has gained traction as a means to remotely diagnose and manage skin conditions. This approach relies heavily on mobile applications, allowing patients to snap pictures of their skin issues using everyday devices like smartphones and tablets. These images are then sent to dermatologists for assessment, eliminating the need for in-person appointments. \par
\vspace{\baselineskip}
However, the success of teledermatology depends heavily on the quality of the images patients capture. Despite the convenience of modern technology, factors like poor lighting, blurry pictures, and unclear depiction of skin problems can make it difficult for dermatologists to give accurate diagnoses. As a result, they face challenges in interpreting these subpar images, which hampers their ability to provide accurate remote diagnoses. \par
\vspace{\baselineskip}
Furthermore, it's important to note that many images submitted by patients don't meet the required standards. This widespread issue highlights the urgent need to improve the clarity and accuracy of images captured through mobile applications. \par 

introduce importance of automated IQA, particularly in teledermatology \par 
highlight recent advancements in AI and its applicaiton in image analysis, including role of deep learning \par
identify challenges associated with manual image quality asessment in TD and the impact of poor image quality on diagnosis accuracy \par

\vspace{\baselineskip}
\noindent

\section{Objectives of the Thesis}
\label{sec:Objectives}
The primary aim of this thesis is to develop and assess automated methods for evaluating image quality in the context of teledermatology. Specific goals include conducting a comprehensive literature review on image quality assessment (IQA) methods in the general image domain and exploring their applicability to teledermatology. Furthermore, the objectives encompass selecting appropriate quality assessment metrics, evaluating these methods using relevant dermatology datasets, and establishing a reproducible repository. \par
\vspace{\baselineskip}
In detail, the objectives are as follows:
\begin{itemize}
    \item Literature Review: Conduct an extensive review of state-of-the-art image quality assessment methods, focusing on their applicability to teledermatology. This review will serve as the foundation for developing robust quality assessment techniques tailored to dermatological images.
    \item Identification of Image Quality Criteria: Identify and delineate specific image quality criteria relevant to the accurate diagnosis of skin conditions in teledermatology. This step is crucial for establishing benchmarks and guidelines for assessing image quality in dermatological contexts.
    \item Evaluation of Methods: Evaluate selected quality assessment methods on publicly available dermatology datasets. This evaluation process will involve assessing the efficacy and accuracy of these methods in objectively quantifying image quality.
    \item Development of a Reproducible Repository: Create a well-documented and reproducible repository that facilitates the replication of reported results and enables the assessment of image quality for new patient images. This repository will serve as a valuable resource for researchers and practitioners in the field of teledermatology.
\end{itemize}
Achieving these objectives is expected to enhance the efficiency and accuracy of teledermatology by establishing a standardized process for assessing image quality. This, in turn, will streamline the workflow in teledermatology, providing robust tools and methodologies for assessing the quality of patient images. Ultimately, these advancements will contribute to improved diagnostic accuracy and patient care in remote dermatological consultations. \par

\section{Structure of the Thesis}
\label{sec:Structure}
\todo{After finishing Chapter Results and Analysis, write this section.}
text \par
\vspace{\baselineskip}
\noindent
